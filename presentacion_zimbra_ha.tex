% $Header: /cvsroot/latex-beamer/latex-beamer/solutions/generic-talks/generic-ornate-15min-45min.en.tex,v 1.4 2004/10/07 20:53:08 tantau Exp $

\documentclass[10pt]{beamer}

\definecolor{verdfosc}{rgb}{0,0.2,0}


\newsavebox{\savepar}
\newenvironment{boxit}%
{\begin{lrbox}{\savepar}
  \begin{minipage}[b]{20cm}}
{\end{minipage}\end{lrbox}\fbox{\usebox{\savepar}}}


\newenvironment{quadre}[1]%
{\begin{center}\begin{lrbox}{\savepar}
  \begin{minipage}[b]{#1 cm}}
{\end{minipage}\end{lrbox}\fbox{\usebox{\savepar}}\end{center}}

\newenvironment{codi}[1]%
{\begin{small}
 \color{verdfosc}
 \begin{quadre}{#1}  
 

 }
{ 
  \end{quadre}
  \end{small} 
  \normalcolor
}




\mode<presentation>
{
  \usetheme{Warsaw}
  % or ...Warsaw Malmoe, Singapore, PaloAlto, Warsaw, Copenhagen

  \setbeamercovered{transparent}
  % or whatever (possibly just delete it)
}


\usepackage[english]{babel}
\usepackage[latin1]{inputenc}
\usepackage{color}
\usepackage{listings}
%Listings setup:
\definecolor{grey}{rgb}{0.8,0.8,0.8}

\lstloadlanguages{Java,XML}

\lstset{language=Java,
        numbers=left, 
        numberstyle=\footnotesize, 
        numbersep=7pt,
        frame=shadowbox, 
        framexleftmargin=5mm, 
        xleftmargin=0.6cm,
        rulesepcolor=\color{grey}
       }







%\usepackage{pstricks,pst-node,pst-tree}
%\usepackage{times}
%\usepackage[T1]{fontenc}
% Or whatever. Note that the encoding and the font should match. If T1
% does not look nice, try deleting the line with the fontenc.

\title[Zimbra 8 High Availability on Ubuntu 12.04]{Zimbra 8 High Availability on Ubuntu 12.04}

%\subtitle{Free as in freedom}

\author[Adri\'an Gibanel L\'opez] % (optional, use only with lots of authors)
{Adri\'an Gibanel L\'opez}
%{F.~Author\inst{1} \and S.~Another\inst{2}}
% - Use the \inst{?} command only if the authors have different
%   affiliation.

\institute[Universitat de Lleida] % (optional, but mostly needed)
{Universitat de Lleida}

%  \inst{1}%
%  Department of Computer Science\\
%  University of Somewhere
%  \and
%  \inst{2}%
%  Department of Theoretical Philosophy\\
%  University of Elsewhere}
% - Use the \inst command only if there are several affiliations.
% - Keep it simple, no one is interested in your street address.

\date[]{September 2013}

%\subject{Cooperative Game Theory}
% This is only inserted into the PDF information catalog. Can be left
% out. 



% If you have a file called "university-logo-filename.xxx", where xxx
% is a graphic format that can be processed by latex or pdflatex,
% resp., then you can add a logo as follows:

% \pgfdeclareimage[height=0.5cm]{university-logo}{university-logo-filename}
% \logo{\pgfuseimage{university-logo}}



% Delete this, if you do not want the table of contents to pop up at
% the beginning of each subsection:
\AtBeginSubsection[]
{
  \begin{frame}<beamer>
    \frametitle{Outline}
    \tableofcontents[currentsection,currentsubsection]
  \end{frame}
}


% If you wish to uncover everything in a step-wise fashion, uncomment
% the following command: 

%\beamerdefaultoverlayspecification{<+->}


\begin{document}

\definecolor{un}{rgb}{0.75,0,0}
\definecolor{dos}{cmyk}{0.95,0,0.5,0.3}
\definecolor{tres}{rgb}{0.85,0.75,0}
\definecolor{quatre}{rgb}{0.2,0.4,0.9}


\setbeamercolor{cyanBox}{fg=black,bg=cyan}

\begin{frame}
  \titlepage
\end{frame}


%\section*{Outline}
  \begin{frame}<beamer>
    \frametitle{Outline}
    \tableofcontents[section]
  \end{frame}


\section {Zimbra}
\begin{frame}
\frametitle{Zimbra Collaboration Server}


\begin{block}{}
Complete collaboration solution accessible from web client, offline clients and other emails and mobile devices
\end{block}


\quad

\begin{itemize}

\item Email
\item Address Book
\item Calendar
\item Tasks

\end{itemize}

\end{frame}

\begin{frame}
\frametitle{Zimbra Collaboration Server}



\end{frame}
\section {Zimbra High Availability}
\section {HA Schema}
\section {LAB}
%\subsection {Operating System Installation}
%\subsection {Network setup}
%\subsection {Zimbra installation}
%\subsection {DRBD Setup}
%\subsection {Disable Zimbra and DRBD startup scripts}
%\subsection {Corosync setup}
%\subsection {Zimbra OCF Resource Agent Development}
%\subsection {Pacemaker setup}
%\section {HA System Management}
\section {Live DEMO}
\section {Future work}
\section {Conclusions}



%\section{Section1}


\begin{frame}[fragile]
\frametitle{Section1FrameTitle}

\begin{block}{Block Title}
This is inside a block.


\end{block}

Outside the block.

\end{frame}

\begin{frame}
\frametitle{Another frametitle inside Section1}


\begin{block}{}
Inside the block.
\end{block}


\quad


After quad:

\begin{itemize}

\item {\tt Remarqued item}

Non remarqued text

{\tt Remarqued text} 

 \item {\tt Another remarqued item}

Non remarqued text 2 

\end{itemize}

\end{frame}
%*****************************************************


%\section{Section2}


\begin{frame}[fragile]
\frametitle{Section2FrameTitle}

\begin{block}{Block Title}
This is inside a block.


\end{block}

Outside the block.

\end{frame}

\begin{frame}
\frametitle{Another frametitle inside Section1}


\begin{block}{}
Inside the block.
\end{block}


\quad


After quad:

\begin{itemize}

\item {\tt Remarqued item}

Non remarqued text

{\tt Remarqued text} 

 \item {\tt Another remarqued item}

Non remarqued text 2 

\end{itemize}

\end{frame}
%*****************************************************




%-----------------------------------------------------

%-----------------------------------------------------

\begin{frame}

\frametitle{Thank you!}


\begin{center}
Thank you!
\end{center}




\end{frame}

%-----------------------------------------------------
\end{document}
